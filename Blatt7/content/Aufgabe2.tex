\section{Aufgabe 2}
\label{sec:Aufgabe2}
% \lstinputlisting[language=Python, firstline=15, lastline=21]{plots/plot.py}
\subsection{a)}
Die Loss Funktion beschreibt den Schaden, der bei einem Entscheidungsproblem entsteht.
Dabei ordnet sie jeder Entscheidung einen Schaden zu, welcher durch die Abweichung von dem wahren Parameter entsteht.
Eine Loss Funktion
\begin{equation}
  L[Y, f(x)]
\end{equation}
wird durch die wahren Outputparameter $Y$ und die aus den Startparametern $x$ gefundene Funktion $f(x)$ beschrieben.
Die Funktion $f(x)$ gibt dabei eine Approximation für die Outputparameter an.

\subsection{b)}
Least Square Method, Support Vector Machines, uvm.
Bei neuronalen Netzen wird meistens das Gradientenabstiegsverfahren angewendet.
\subsection{c)}
Aktivierungsfunktionen sind nichtlinear und dienen dazu den Raum bei der Transformation so zu verzerren, dass Daten im verzerrten Raum mit geraden Schnitten getrennt werden können.
Ohne die Aktivierungsfunktion wäre ein Neuronales Netz unsinnvoll, weil weiterhin einfach eine Reihe von linearen Transformationen durchgeführt wird.\\
Gängige Aktivierungsfunktionen sind z.B. der Tangens-Hyperbolicus-Funktion, die ReLU(Rectified-Linear-Unit)-Aktivierungsfunktion
\begin{equation}
  f(x) = max(0,x)
\end{equation}
und die Leaky-ReLU-Funktion
\begin{equation}
  f(x) = \symbb{1}(x<0)(\alpha x) + \symbb{1}(x\geq0)(x)
\end{equation}

\subsection{d)}

Die Zahl der Neuronen ist die Summe der Hidden-Layer und Output-Layer also alle Layer zu denen Informationen hingehen und verarbeitet werden.

\subsection{e)}
Bildsignalverarbeitung,
Trennen von Kreisförmig angeordneten Daten
