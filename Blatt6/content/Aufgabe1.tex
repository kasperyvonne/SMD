\section{Aufgabe1}
\label{sec:Aufgabe1}
%\lstinputlisting[language=Python, firstline=15, lastline=21]{plots/plot.py}
\paragraph{a)} 
Sind die Populationen unterschiedlich stark vertretten neigt der KNN Algorithmus dazu die 
Elemente ehr in die stärker vertrettene Population einzuordnen, da es immer mehr k-Nachbarn von  
der größeren Population geben wird.  
\paragraph{b)} Der Algorithmus wird als \enquote{lazy learner} bezeichnet, da ein Teil des Lehrprozesses 
beinhaltet, dass der Trainingsdatensatz teilweise abgespeichert wird. Das wir in der Literatur als 
\enquote{lazy learning} bezeichnet. Im Vergleich zu anderen Algorithmus sehr kosten günstig, da 
er nur auf jedes Element des zu analysierenden Datensatzes angewendet werden muss (N Iterationen). 
Dabei spielt es keine Rolle in wieviel Dimensionen das Problem analysiert werden muss, 
so lange es ein geeignetes Abstandsmaß gibt. Genau an dieser Stelle sticht der KNN den SVM 
aus, da die Recheneffiziens des SVM stark von der Dimension des Problems abhängt.
\paragraph{c)}\quad \newline
\lstinputlisting[language=Python, firstline=3, lastline=65]{plots/class_structure.py}
\paragraph{d,e,f)}
Werte einlesen, Trainingsdatenset erstellen, Datenset erstellen:
\lstinputlisting[language=Python, firstline=66, lastline=84]{plots/class_structure.py} 	
Werte nach folgender Form durch den KNN jagen (einmal für d gezeigt):
\lstinputlisting[language=Python, firstline=86, lastline=89]{plots/class_structure.py}
Funktion die Reinheit, Effizienz und Signifikanz bestimmt (Werte dazu in \ref{tab:tab}):
\lstinputlisting[language=Python, firstline=100, lastline=112]{plots/class_structure.py}
In d) sieht man gut was in a) diskutiert wird, da der Untergrund doppelt so groß ist wie das 
Signal werden nur noch ca. ein Fünftel richtig zu geordnet. 
Das erklärt warum Reinheit und Effizienz klein ausfallen. 
Bei der Signifikanz ist iwas schief gelaufen. \\
Bei der e) sieht man, dass die Anzahl der Hits wesentliches Attribut zur Unterscheidung darstellt. 
Der Logarithmus sorgt dafür, dass die Werte dahingehend näher zusammenrücken und können nicht mehr 
unterschieden werden vom Untergrund. \\
In der f) tritt dann, das so genannte Overfitting ein, $k = 10$ scheint schon ein guter  
Wert zu seien, um beide Population zu trennen. $k = 20$ dagegen scheint dagegen ins Overfitting 
zu gehen, da Reinheit, Effizienz (und Signifikanz) abnehmen. 
\begin{table}
 \centering
 \begin{tabular}{c|ccc}
   \toprule
	Teilaufgabe & Reinheit & Effizienz & Signifikanz \\
   \midrule
	d)&0.0921& 0.1842& 13.94878250362136\\
	e)&0.0& 0.0& 0.0\\
	f)&0.08535& 0.1707& 13.290603196745186\\
   \bottomrule
 \end{tabular}
 \caption{Reinheit, Effizienz und Signifikanz für die einzelnen teilaufgaben.}
 \label{tab:tab}
\end{table}
                              
