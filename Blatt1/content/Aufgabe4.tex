\section{Aufgabe 4}
\label{sec:Aufgabe4}
\subsection{Teil a)}
Die Wahrscheinlichkeit, dass die Summe der Würfel 9 ist, lässt sich aus den verschiedenen Möglichkeiten den
erwünschten Wert 9 aus allen Ereignissen zu erhalen,
bestimmen. Insgesamt gibt es 36 mögliche Ereignisse. Mit zwei Würfeln lässt sich der Wert 9 nur mit den Kombinationen 3\&6 sowie 4\&5 erreichen.
Insgesamt gibt es also 4 Möglichkeiten, da es egal ist welcher der Würfel, welchen Wert hat.
Daraus folgt
\begin{equation*}
  P(W_{\symup{rot}} + W_{\symup{blau}} = 9) = \frac{4}{36} = \frac{1}{9} .
\end{equation*}
\subsection{Teil b)}
Um die Wahrscheinlichkeit zu bestimmen, eine Summe von 9 oder mehr zu erhalten,
werden die gewünschten Kombinationen abgezählt: 6\&3, 3\&6, 6\&4, 4\&6, 6\&5, 5\&6, 6\&6, 5\&4, 4\&5, 5\&5.
Es gibt also insgesamt 10 Möglichkeiten.
\begin{equation*}
  P(W_{\symup{rot}} + W_{\symup{blau}} \geq 9) = \frac{10}{36} = \frac{5}{18} .
\end{equation*}
\subsection{Teil c)}
Es gibt nur 2 Möglichkeiten, dass ein Würfel 4 und der andere 5 zeigt.
\begin{equation*}
  P(W_{\symup{blau/rot}} = 4, W_{\symup{blau/rot}} = 5) = \frac{2}{36} = \frac{1}{18} .
\end{equation*}
\subsection{Teil d)}
Es gibt genau eine Möglichkeit, dass der rote Würfel 4 und der blaue 5 zeigt.
\begin{equation*}
  P(W_{\symup{rot}} = 4, W_{\symup{blau}} = 5) = \frac{1}{36} .
\end{equation*}
\subsection{Teil e)}
Da bekannt ist, dass der rote Würfel eine 4 zeigt, bleiben noch 6 mögliche Ereignisse
für den blauen Würfel.
Dass die Summe 9 ist gilt also nur, wenn der blaue Würfel eine 5 zeigt.
\begin{equation*}
   P(W_{\symup{rot}} + W_{\symup{blau}} = 9 |W_{\symup{rot}} = 4) = P( W_{\symup{blau}} = 5) =  \frac{1}{6} .
\end{equation*}
\subsection{Teil f)}
Es gibt nur zwei Ereignisse, 5 und 6, die zusammen mit der roten 4 eine 9 ergeben:
\begin{equation*}
  P(W_{\symup{rot}} + W_{\symup{blau}} \geq 9 \ | \ W_{\symup{rot}} = 4) = \frac{2}{6} = \frac{1}{3}.
\end{equation*}
\subsection{Aufgabenteil g)}
Es gibt nur eine Möglichkeit, also gilt wie in Teil e) :
\begin{equation*}
  P(W_{\symup{rot}} = 4, \, W_{\symup{blau}} = 5 \ | \ W_{\symup{rot}} = 4) = \frac{1}{6}.
\end{equation*}
