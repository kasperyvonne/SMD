\section{Aufgabe 5}
\label{sec:Aufgabe5}
%\lstinputlisting[language=Python, firstline=15, lastline=21]{plots/plot.py}
\paragraph{a} Die Nullhypotese hier besagt, dass der Bin $i$ den Erwartungswert $\mu_i$ besitzt. Die 
Zahl der Eintrage $n_i$ im Bin $i$ sind Zufallsvariablen verteilt gemäß der Poisson-Verteilung. 
Damit folgt
\begin{gather}
P(n_i \vert \mu_i) = \frac{\mu_{i}^{ n_i} \exp(-\mu_i)}{n_i !} 	
\quad \text{für} \quad n_i \\
P(m_i \vert \mu_i) = \frac{\mu_{i}^{ m_i} \exp(-\mu_i)}{m_i !} 	
\quad \text{für} \quad m_i 
\end{gather}
die Wahrscheinlichkeitsdichte für ein Eintrag in einem Bin. 

\paragraph{b}
Hier bietet es sich an die negative Log-Likelyhood Funktion zu verwenden. 
\begin{equation}
\begin{split}
F(\mu_i) &= - \sum_{i=1}^r \log(P(n_i \vert \mu_i)\cdot P(m_i \vert \mu_i))\\
&= - \sum_{i=1}^r \log \left(\frac{\mu_{i}^{ n_i} \exp(-\mu_i)}{n_i !} \right)
-\sum_{i=1}^r \log \left(\frac{\mu_{i}^{ m_i} \exp(-\mu_i)}{m_i !} \right)\\
&= - \sum_{i=1}^r (n_i+m_i)\log(\mu_{i})  +2\sum_{i=1}^r\mu_i + 
\underbrace{\sum_{i=1}^r \log(n_i !\cdot m_i!)}_{ =konst  \text{, da $\mu_i$ unabhängig}}
\end{split}
\end{equation}
Die Likelyhood muss nun minimiert werden,deshalb
\begin{gather}
\frac{d F(\mu_i)}{d\mu_i} \stackrel{!}{=} 0 \\
\frac{d F(\mu_i)}{d\mu_i} =  - \sum_{i=1}^r \frac{(n_i + m_i)}{\mu_{i}}  +2\sum_{i=1}^r 1  \\
\implies - \frac{1}{\mu_{i}}\sum_{i=1}^r (n_i + m_i) + 2r = 0 \\
\implies \hat{\mu_i} = \frac{\sum_{i=1}^r  (n_i + m_i)}{2r} = \frac{N + M}{2r}
\end{gather}

\paragraph{c}
Der $\chi^2$-Test ist gegeben über
\begin{equation}
 t = \chi^2 = \sum_{i=1}^r \frac{(n_i-n_0)^2}{n_0}+\sum_{i=1}^r \frac{(m_i-m_0)^2}{m_0} 
\end{equation}
mit $n_0 = \frac{N}{r}$ und $m_0 = \frac{M}{r}$.

\paragraph{d} 
Die $\chi^2$-Verteilung hat $k=r$ Freiheitsgerade, vorrausgesetzt die Einträge sind nicht normiert. 
Dann wäre der Freiheitsgerade $k= r-1$.\\
Für den hier verwendeten $\chi^2$-Test wird die Gaußsche-Nährung $\sigma^2 = n_0$ verwendet, diese 
ist aber nur für $n_o$ hinreichend groß $(>10)$ erfüllt. Ist $n_0 < 10$ kann dem Test nicht mehr vertraut 
werden, da er nicht mehr einer $\chi^2$-Verteilung folgt.
\paragraph{e}
Für den angegeben $\chi^2$-Test ergibt sich $t\approx 8.43$.
Nun werden die Signifikanzen $\alpha = 0,1 ; 0,05 ; 0,01$ betraachtet, damit ergibt sich 
$(1-\alpha) = 0,9 ; 0,95 ; 0,99$. Die dazu gehörigen $\chi^2$-Werte für $k = 3$ sind 
$\chi^2(3) = 6,25;7,81;	11,34$ damit würde die Nullhypotese nur für $\alpha=0,01$ nicht verworfen werden. 

